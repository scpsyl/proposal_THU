% !TeX root = ../thuthesis-example.tex

\begin{resume}

  \section*{个人简历}

  2003 年 9 月 11 日出生于吉林省蛟河市。

  2020 年 9 月考入吉林大学电子科学与工程学院电子信息科学与技术专业,2024 年 6 月本科毕业并获得理学学士学位。

  2024 年 9 月考入清华大学深圳国际研究生院,攻读电子与通信工程硕士至今。


  \section*{在学期间完成的相关学术成果}

  \subsection*{学术论文}

  \begin{achievements}
    \item \textbf{Liu Y}, Mu S, Chao X, et al. AVR: Active Vision-Driven Precise Robot Manipulation with Viewpoint and Focal Length Optimization.
    \item Chao X, Mu S, \textbf{Liu Y}, et al. Exo-ViHa: A Cross-Platform Exoskeleton System with Visual and Haptic Feedback for Efficient Dexterous Skill Learning.
    \item Li S, \textbf{Liu Y}, Chao X, et al. ALARMbot: Autonomous Laboratory Safety Inspection and Operable Hazard Intervention Robot Enabled by Foundation Models.
  \end{achievements}


  % \subsection*{专利}

  % \begin{achievements}
  %   \item 任天令, 杨轶, 朱一平, 等. 硅基铁电微声学传感器畴极化区域控制和电极连接的方法: 中国, CN1602118A[P]. 2005-03-30.
  %   \item Ren T L, Yang Y, Zhu Y P, et al. Piezoelectric micro acoustic sensor based on ferroelectric materials: USA, No.11/215, 102[P]. (美国发明专利申请号.)
  % \end{achievements}

\end{resume}



% 本科生格式:

% \begin{resume}
%   \section*{学术论文}
%
%   \begin{achievements}
%     \item ZHOU R, HU C, OU T, et al. Intelligent GRU-RIC Position-Loop
%       Feedforward Compensation Control Method with Application to an
%       Ultraprecision Motion Stage[J], IEEE Transactions on Industrial
%       Informatics, 2024, 20(4): 5609-5621.
%
%     \item 杨轶, 张宁欣, 任天令, 等. 硅基铁电微声学器件中薄膜残余应力的研究[J].
%       中国机械工程, 2005, 16(14):1289-1291.
%
%     \item YANG Y, REN T L, ZHU Y P, et al. PMUTs for handwriting recognition.
%       In press[J]. (已被Integrated Ferroelectrics录用)
%
%   \end{achievements}
%
%
%   \section*{专利}
%
%   \begin{achievements}
%     \item 胡楚雄, 付宏, 朱煜, 等. 一种磁悬浮平面电机: ZL202011322520.6[P]. 2022-04-01.
%
%     \item REN T L, YANG Y, ZHU Y P, et al. Piezoelectric micro acoustic sensor
%       based on ferroelectric materials: No.11/215, 102[P]. (美国发明专利申请号.)
%
%   \end{achievements}
% \end{resume}
