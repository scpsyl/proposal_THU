% !TeX root = ../thuthesis-example.tex

\chapter{课题主要研究内容}

\section{拟解决的关键问题}

当前具身移动操作方法在通用场景下面临以下瓶颈:

\begin{enumerate}
  \item 现有 VLA 模型多以“从图像与文本直接预测动作序列”的方式工作,缺少能够直接反映操作位置与姿态的中间表示。
  在复杂三维环境中,因缺乏显式可达性与安全约束建模,易出现物理不可达或存在碰撞风险的策略。

  \item 多数系统沿用“先导航、后操作”的弱耦合范式。对于视角与姿态高度敏感的任务,这种设计难以同时兼顾视角质量、末端可达域和运动安全性,
  从而对任务成功率与执行效率形成明显制约。

  \item 高质量移动操作数据的采集与标注成本较高,不同场景和机器人差异显著,在新环境中的性能急剧下降。
  尽管已有跨平台数据集和高保真仿真环境,如何有效利用异构数据与模拟数据,实现对通用室内场景的稳健泛化,缺乏成熟的方法体系。
\end{enumerate}

\section{拟研究内容}

\begin{enumerate}
  \item 在 VLA 框架下,探索将语言意图、场景语义结构、几何关系与动作可达性等信息统一编码的中间表征形式,形成适配移动操作任务的视觉编码器、语言主干与动作输出层设计,
  为后续协同决策与安全约束提供可解释的语义–几何桥接层。

  \item 构建能够同时处理多模态观测、兼顾全局移动与局部精细操作需求的协同决策框架,在同一策略体系内协调基座位姿选择、视角配置与末端动作规划,从而在复杂室内环境中提升整体任务执行的稳定性与效率。

  \item 基于典型通用室内场景,构建覆盖多类语言驱动移动操作任务的统一任务集,设计兼顾任务成功率、路径与操作效率以及跨场景迁移能力的评价指标;
  在仿真与真实机器人平台上开展系统集成与对比实验,验证所提出 VLA 移动操作方法在多任务、多场景条件下的有效性与适用范围。
\end{enumerate}
