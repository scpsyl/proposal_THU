% !TeX root = ../thuthesis-example.tex

\chapter{课题主要研究内容}

\section{拟解决的关键问题}

当前具身移动操作方法在通用场景下面临以下瓶颈:

\begin{enumerate}
  \item 现有 VLA 模型多以“从图像与文本直接预测动作序列”的方式工作,缺少能够直接反映操作位置与姿态的中间表示。
  在复杂三维环境中,因缺乏显式可达性与安全约束建模,易出现物理不可达或存在碰撞风险的策略。

  \item 多数系统沿用“先导航、后操作”的弱耦合范式。对于视角与姿态高度敏感的任务,这种设计难以同时兼顾视角质量、末端可达域和运动安全性,
  从而对任务成功率与执行效率形成明显制约。

  \item 高质量移动操作数据的采集与标注成本较高,不同场景和机器人差异显著,在新环境中的性能急剧下降。
  尽管已有跨平台数据集和高保真仿真环境,如何有效利用异构数据与模拟数据,实现对通用室内场景的稳健泛化,缺乏成熟的方法体系。
\end{enumerate}

\section{拟研究内容}

\begin{enumerate}
  \item 围绕具身移动操作任务,在 VLA 框架下设计统一的语义--几何中间表征形式,
  将语言意图、场景语义结构、几何可达性与安全约束等信息显式对齐,
  完成相应的视觉编码器、语言主干与动作输出层实现,为后续一体化策略与安全分析提供可直接调用的表征基础。

  \item 基于前述中间表征,构建融合主动视觉与全身控制的移动--操作协同决策框架,
  在同一策略体系内联合建模基座位姿选择、视角/焦距配置与单/双臂末端动作规划,
  并在典型实验室及类家庭场景中验证其在复杂几何约束与风险约束下的稳定性与鲁棒性。

  \item 面向实验室、家居与仓储等典型通用室内场景,
  构建覆盖多类语言驱动移动操作任务的统一任务集与评价指标体系,
  结合高保真仿真环境与真实轮式双臂平台开展系统集成与对比实验,
  系统评估所提出 VLA 移动操作方法在多任务、多场景及跨平台条件下的有效性与泛化能力。
\end{enumerate}

