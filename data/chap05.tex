% !TeX root = ../thuthesis-example.tex

\chapter{预期创新点与研究成果}

\section{预期创新点}

\begin{enumerate}
  \item 提出显式编码语言意图、场景语义结构、几何关系与动作可达性约束的 VLA 中间表征与模型架构,
  将传统“黑箱式”从图像/文本到动作的映射,提升为可解释的语义--几何桥接层,为安全约束和行为验证提供基础。

  \item 在统一策略框架下,将基座位姿选择、视角配置与末端动作规划纳入同一优化过程,形成兼顾全局移动与局部精细操作需求的协同决策方法。

  \item 面向通用室内场景构建语言驱动移动操作任务集与评价指标体系,结合异构真实数据和高保真仿真数据,探索数据高成本条件下的 VLA 训练与跨场景迁移策略。
\end{enumerate}

\section{预期成果}

完成基于真实移动操作本体的原型系统集成与示范应用,在实验室场景和企业应用场景中验证所提方法的工程可行性,并在机器人与人工智能等方向的高水平国际会议或期刊上发表若干篇学术论文。




