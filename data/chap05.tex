% !TeX root = ../thuthesis-example.tex

\chapter{预期创新点与研究成果}

\section{预期创新点}

\begin{enumerate}
  \item 提出面向真实移动操作的语义--几何一体化 VLA 中间表征,
  显式编码语言意图、场景语义结构、几何可达性与安全约束,将实验室安全巡检等任务抽象为统一表示,
  为可解释、可约束的导航--操作一体化决策提供基础。

  \item 构建融合主动视觉与全身控制的移动--操作协同决策机制,在统一策略框架下联合优化基座位姿、视角/焦距配置与双臂末端动作,
  结合扩散策略与自回归全身动作解码,提升在狭窄、遮挡和高风险区域内精细操作的稳定性与鲁棒性。

  \item 面向通用室内场景构建基于轮式双臂平台的语言驱动移动操作 Benchmark 与评价体系,结
  合 BEHAVIOR-1K 等高保真仿真环境与真实 VR 遥操作数据,形成兼顾数据成本与跨场景泛化的 VLA 训练与评测方法,
  为不同算法提供可复现、可对标的统一基准。
\end{enumerate}


\section{预期成果}

完成基于真实移动操作本体的原型系统集成与示范应用,在实验室场景和企业应用场景中验证所提方法的工程可行性,
并在机器人、具身智能、人工智能等方向的高水平国际会议或期刊上发表若干篇学术论文。




